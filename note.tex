\section{Lucas's theorem}
\[{m\choose n}\equiv\prod_{i=0}^k{m_i \choose n_i}(\mbox{mod }p)\] where $m=m_kp^k+m_{k-1}p^{k-1}+\cdots+m_1p+m_0$ and 
$n=n_kp^k+n_{k-1}p^{k-1}+\cdots+n_1p+n_0$.

\section{無權邊的生成樹個數 Kirchhoff's Theorem}
1. 定義 $n\times m$矩陣 $E=(a_{i,j})$,$n$ 為點數, $m$ 為邊數,若 $i$ 點在 $j$ 邊上,$i$為小點 $a_{i,j}=1$, $i$為大點 $a_{i,j}=-1$,否則 $a_{i,j}=0$。\\
(證明省略)
\\
4.令 $E(E^T)=Q$,他是一種有負號的 kirchhoff 的矩陣,取 $Q$ 的子矩陣即為 $F(F^T)$\\
結論:做 $Q$ 取子矩陣算 det 即為所求。(除去第一行第一列 by mz)

\section{monge}
$i \leq i^{'} < j \leq j^{'}$ \\
$m(i,j)+m(i^{'},j^{'}) \leq m(i^{'},j)+m(i,j^{'})$ \\
$k(i,j-1)<=k(i,j)<=k(i+1,j)$

\section{四心}
$\frac{sa*A+sb*B+sc*C}{sa+sb+sc}$ \\
外心 sin 2A : sin 2B : sin 2C \\
內心 sin  A : sin  B : sin  C \\
垂心 tan  A : tan  B : tan  C \\
重心      1 :      1 :      1 

\section{Runge-Kutta}
$y_{n+1}=y_n+\frac{h}{6}(k_1+2k_2+2k_3+k_4)$\\
$k_1=f(t_n,y_n)$\\
$k_2=f(t_n+\frac{h}{2},y_n+\frac{h}{2}k_2)$\\
$k_3=f(t_n+\frac{h}{2},y_n+\frac{h}{2}k_3)$\\
$k_2=f(t_n+h,y_n+hk_3)$

\section{Householder Matrix}
$I-2\frac{vv^T}{v^Tv}$